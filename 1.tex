\documentclass[12pt]{article}
\usepackage{ctex}
\usepackage{geometry}
\usepackage{graphicx}
\usepackage{amsmath}
\usepackage{amssymb}
\usepackage{color}
\usepackage{hyperref}
\usepackage{titlesec}
\usepackage{fancyhdr}
\usepackage{indentfirst}
\usepackage{float} % 放在导言区


\geometry{a4paper, margin=2.5cm}
\setlength{\parindent}{2em} % 首行缩进
\pagestyle{fancy}
\fancyhf{}
\rhead{卫星波束复用控制系统设计与实现}
\lhead{Mikuma}
\cfoot{\thepage}

\titleformat{\section}{\bf\large}{\thesection}{1em}{}
\titleformat{\subsection}{\bf}{\thesubsection}{1em}{}

\title{卫星波束复用控制系统设计与实现}
\author{杨一}
\date{\today}

\begin{document}

\maketitle

\begin{abstract}
本文提出并实现了一种基于Python语言和Tkinter图形界面库的简化模拟卫星移动通信系统。系统能够模拟信号输入的数模转换过程,设计了一套自创的编码系统与通信协议,可以模拟通信传输,解调与调制过程。可以实现对移动通信交换子系统的模拟,此外,还完成了模拟多波束数量配置复用、模拟功率调节、频率资源优化分配及干扰预警功能。具备较强的实时交互能力和系统可扩展性。通过Matplotlib实现了以上功能的动态可视化,为用户提供直观的波束管理界面。系统设计过程强调模块化架构和多线程事件响应机制,保证了系统的稳定运行和良好的用户体验。通过多轮实验验证,系统在波束配置准确性、协议模拟准确性,通信模块传输模拟真实性,模块调节响应速度等方面表现良好。
\end{abstract}

\section{各模块设计流程简要描述}
本系统设计流程严格遵循软件工程模块化开发理念,整体架构划分为四个主要模块:ADDC模拟模块、通信编码与协议实现与解译模块、模拟传输,调制解调模块、波束复用与管理模块。设计过程从用户需求分析出发,逐步细化各功能模块,确保各模块职责单一且接口清晰,利于后期维护与升级。




\subsection{ADDC模拟模块设计说明}

\subsubsection{功能概述}
该模块实现了卫星通信系统中的模拟/数字信号转换功能,主要包含以下核心功能:
\begin{itemize}
    \item 模拟信号生成与显示(正弦波、方波、三角波)
    \item AD转换与数字量化
    \item PWM调制与解调
    \item 信号重建与滤波
\end{itemize}

\subsubsection{系统架构设计}
系统采用三区块设计结构:
\begin{enumerate}
    \item \textbf{信号生成区}
    \begin{itemize}
        \item 固定频率10kHz波形生成
        \item 可调节信号幅值(0.1-10.0V)
        \item 支持多种波形选择与参数调节
    \end{itemize}
    
    \item \textbf{AD转换区}
    \begin{itemize}
        \item 固定8位量化精度
        \item 可调采样率(0.1-5000kHz)
        \item 可配置参考电压(1.0-20.0V)
        \item PWM调制(0.05-5.0kHz)
    \end{itemize}
    
    \item \textbf{信号重建区}
    \begin{itemize}
        \item 数字滤波实现
        \item 支持移动平均与巴特沃斯滤波
        \item 实时信号质量监测(RMSE与SNR)
    \end{itemize}
\end{enumerate}

\subsubsection{关键技术实现}
\paragraph{AD转换过程}
AD转换采用8位量化,通过以下步骤实现:
\begin{equation}
D_{out} = round(\frac{V_{in} + V_{ref}}{2V_{ref}} \times (2^8-1))
\end{equation}

其中:
\begin{itemize}
    \item $V_{in}$:输入电压
    \item $V_{ref}$:参考电压
    \item $D_{out}$:输出数字量
\end{itemize}

\paragraph{PWM调制技术}
采用基于占空比调制的PWM技术:
\begin{equation}
PWM_{duty} = \frac{D_{in}}{2^8-1} \times 100\%
\end{equation}

其中$D_{in}$为输入数字量。

\subsubsection{性能指标}
系统主要性能指标如下:
\begin{itemize}
    \item 采样率范围:0.1-5000kHz
    \item 信号幅值范围:0.1-10.0V
    \item PWM频率范围:0.05-5.0kHz
    \item 量化精度:8位
    \item 参考电压范围:1.0-20.0V
\end{itemize}

\subsubsection{用户界面设计}
系统提供直观的图形化界面,包含:
\begin{itemize}
    \item 三通道实时波形显示
    \item 参数实时调节控件
    \item 信号质量实时监测
    \item 运行状态控制按钮
\end{itemize}

\subsubsection{系统特色}
\begin{enumerate}
    \item 实时性能优越,支持50ms周期更新
    \item 参数调节灵活,支持实时预览
    \item 信号质量监测全面,包含RMSE与SNR指标
    \item 故障容错设计,滤波器异常时自动切换备用方案
\end{enumerate}





\subsection{通信编码与协议实现与解译功能概述}
该模块实现了卫星通信系统中的Turbo编码功能,主要包含以下核心功能:
\begin{itemize}
    \item 随机数字信号生成(16位)
    \item Turbo编码实现(包含交织、RSC编码)
    \item 实时数据监测与显示
    \item 系统运行状态记录
\end{itemize}

\subsubsection{系统架构设计}
系统采用三层架构设计:
\begin{enumerate}
    \item \textbf{编码器核心层}
    \begin{itemize}
        \item Turbo编码器实现
        \item 交织器实现
        \item RSC编码器实现
    \end{itemize}
    
    \item \textbf{数据处理层}
    \begin{itemize}
        \item 随机数据生成
        \item 数据格式化处理
        \item 编码结果组装
    \end{itemize}
    
    \item \textbf{用户界面层}
    \begin{itemize}
        \item 图形界面展示
        \item 数据流监控
        \item 运行状态显示
    \end{itemize}
\end{enumerate}

\subsubsection{关键技术实现}
\paragraph{交织算法}
采用简单周期交织方案:
\begin{equation}
P_{new} = (3i + 7) \bmod L
\end{equation}
其中:
\begin{itemize}
    \item $P_{new}$:交织后位置
    \item $i$:原始位置
    \item $L$:数据长度
\end{itemize}

\paragraph{RSC编码器}
采用基于状态机的递归系统卷积编码:
\begin{itemize}
    \item 2比特状态寄存器
    \item 单比特奇偶校验输出
    \item 状态更新方程:$S_{new} = ((S_{old} << 1) | bit) \& 0x3$
\end{itemize}

\subsubsection{性能指标}
系统主要性能指标如下:
\begin{itemize}
    \item 输入数据长度:16位
    \item 编码率:1/3(系统位+两组校验位)
    \item 状态数:4状态
    \item 更新周期:5秒/次
\end{itemize}

\subsubsection{用户界面设计}
系统提供图形化界面,包含:
\begin{itemize}
    \item 数据流实时显示
    \item 编码结果可视化
    \item 监控日志展示(最近10条)
    \item 运行状态切换控制
\end{itemize}

\subsubsection{系统特色}
\begin{enumerate}
    \item 自动化运行,支持5秒自动更新
    \item 实时监控功能,可随时开启关闭
    \item 支持手动触发数据更新
    \item 监控日志滚动更新机制
\end{enumerate}



\subsection{模拟传输,调制解调模块说明}

\subsubsection{功能概述}
该模块实现了卫星通信系统中的协议仿真功能,主要包含以下核心功能:
\begin{itemize}
    \item 通信数据帧的生成与发送
    \item 数据帧的分析与解析
    \item BPSK调制与解调
    \item 通信过程实时监控
\end{itemize}

\subsubsection{系统架构设计}
系统采用五区块设计结构:
\begin{enumerate}
    \item \textbf{传输交换模块}
    \begin{itemize}
        \item 多信道管理(4信道)
        \item 传输延迟模拟
        \item 路由历史记录
    \end{itemize}
    
    \item \textbf{帧生成器模块}
    \begin{itemize}
        \item 协议帧构建
        \item 参数配置界面
        \item 校验和计算
    \end{itemize}
    
    \item \textbf{帧分析器模块}
    \begin{itemize}
        \item 协议帧解析
        \item 数据提取显示
        \item 错误检测
    \end{itemize}

    \item \textbf{监控日志模块}
    \begin{itemize}
        \item 实时日志记录
        \item 传输状态监控
        \item 日志导出功能
    \end{itemize}

    \item \textbf{调制解调模块}
    \begin{itemize}
        \item BPSK调制实现
        \item 波形可视化
        \item 信号重建
    \end{itemize}
\end{enumerate}

\subsubsection{关键技术实现}
\paragraph{帧格式定义}
采用固定格式的通信帧结构:
\begin{equation}
Frame = Header_{meow} + Addr_{src} + Addr_{dst} + Type + Payload + Checksum + Footer_{mikuma}
\end{equation}

帧结构组成:
\begin{itemize}
    \item Header: 固定为"meow"的16进制表示
    \item Addresses: 4字节源目地址
    \item Type: 2字节帧类型标识
    \item Payload: 可变长度数据负载
    \item Checksum: 2字节校验和(可选)
    \item Footer: 固定为"mikuma"的16进制表示
\end{itemize}

\paragraph{延迟模拟}
采用随机延迟模拟真实传输:
\begin{itemize}
    \item 单跳延迟范围:0.5-1.0秒
    \item 双跳延迟范围:1.0-2.0秒
\end{itemize}

\subsubsection{性能指标}
系统主要性能指标如下:
\begin{itemize}
    \item 支持信道数:4
    \item 地址空间:16位
    \item 帧类型数:3种
    \item 最大传输单元:无限制
    \item 校验方式:简单校验和
\end{itemize}

\subsubsection{用户界面设计}
系统提供图形化界面,包含:
\begin{itemize}
    \item 多标签页界面布局
    \item 实时状态显示栏
    \item 滚动文本显示区
    \item 参数配置面板
    \item 波形显示画布
\end{itemize}

\subsubsection{系统特色}
\begin{enumerate}
    \item 完整协议栈模拟实现
    \item 直观的数据可视化展示
    \item 灵活的参数配置选项
    \item 详细的监控日志记录
    \item 模块化设计便于扩展
\end{enumerate}


\subsection{波束复用与管理模块说明}

\subsubsection{功能概述}
该模块实现了卫星通信系统中的波束复用控制功能,主要包含以下核心功能:
\begin{itemize}
    \item 多波束配置管理(4/7/12波束)
    \item 波束功率动态调节
    \item 频率资源分配优化
    \item 干扰监测与预警
\end{itemize}

\subsubsection{系统架构设计}
系统采用四区块设计结构:
\begin{enumerate}
    \item \textbf{波束管理模块}
    \begin{itemize}
        \item 波束数量配置(4/7/12)
        \item 波束位置计算
        \item 波束状态管理
    \end{itemize}
    
    \item \textbf{资源控制模块}
    \begin{itemize}
        \item 频率资源分配
        \item 功率级别调节
        \item 资源优化算法
    \end{itemize}
    
    \item \textbf{干扰控制模块}
    \begin{itemize}
        \item 干扰阈值设置
        \item 波束间距检测
        \item 频率冲突分析
    \end{itemize}

    \item \textbf{可视化模块}
    \begin{itemize}
        \item 波束分布图绘制
        \item 实时状态监控
        \item 交互式控制界面
    \end{itemize}
\end{enumerate}

\subsubsection{关键技术实现}
\paragraph{波束位置计算}
采用极坐标分布算法:
\begin{equation}
\begin{split}
x_i &= r \cos(\frac{2\pi i}{N}) \\
y_i &= r \sin(\frac{2\pi i}{N})
\end{split}
\end{equation}

其中:
\begin{itemize}
    \item $N$:波束总数
    \item $i$:波束编号
    \item $r$:归一化半径(0.7)
\end{itemize}

\paragraph{干扰检测算法}
基于欧氏距离的干扰判定:
\begin{equation}
D_{ij} = \sqrt{(x_i-x_j)^2 + (y_i-y_j)^2} < \frac{T}{50}
\end{equation}

其中:
\begin{itemize}
    \item $D_{ij}$:波束i和j之间的距离
    \item $T$:干扰阈值(0-100)
\end{itemize}

\subsubsection{性能指标}
系统主要性能指标如下:
\begin{itemize}
    \item 波束配置:4/7/12波束可选
    \item 频率资源:4个频段
    \item 功率调节范围:0-100\%
    \item 干扰阈值范围:0-100\%
    \item 可视化更新率:实时
\end{itemize}

\subsubsection{用户界面设计}
系统提供图形化界面,包含:
\begin{itemize}
    \item 波束分布可视化画布
    \item 功率控制滑动条
    \item 干扰阈值调节器
    \item 频率资源指示器
    \item 状态监控文本框
\end{itemize}

\subsubsection{系统特色}
\begin{enumerate}
    \item 直观的波束分布可视化
    \item 灵活的波束配置选项
    \item 智能的频率资源优化
    \item 实时的干扰监测预警
    \item 完整的状态监控记录
\end{enumerate}



\section{整体总结}


\subsection{系统架构}
\subsubsection{整体架构}
系统采用模块化设计,四个子系统相互独立但功能互补:
\begin{enumerate}
    \item \textbf{信号处理层}
    \begin{itemize}
        \item AD/DA转换模块:实现模拟/数字信号转换
        \item 信号调制解调功能
    \end{itemize}
    
    \item \textbf{编码传输层}
    \begin{itemize}
        \item Turbo编码模块:实现信道编码
        \item 通信协议模块:实现数据帧处理
    \end{itemize}
    
    \item \textbf{资源控制层}
    \begin{itemize}
        \item 波束管理模块:实现空间复用
        \item 频率分配模块:实现频率复用
    \end{itemize}
\end{enumerate}



\subsection{系统特色}
\begin{enumerate}
    \item \textbf{全面的功能覆盖}
    \begin{itemize}
        \item 从信号处理到资源管理的完整链路
        \item 支持多种通信场景仿真
    \end{itemize}
    
    \item \textbf{灵活的配置选项}
    \begin{itemize}
        \item 参数实时调节
        \item 多种工作模式切换
    \end{itemize}
    
    \item \textbf{友好的用户界面}
    \begin{itemize}
        \item 图形化操作界面
        \item 实时数据可视化
    \end{itemize}
    
    \item \textbf{完善的监控机制}
    \begin{itemize}
        \item 全程运行状态监测
        \item 异常情况实时预警
    \end{itemize}
\end{enumerate}

\section{实验结果}
\subsection{模块一结果}


\begin{figure}[H]
    \centering
    \includegraphics[width=0.3\linewidth]{3in1.jpg}
    \caption{三种模式(方波正弦波三角波)对应界面,左侧为原始信号,中间为ADC生成的数字信号,右侧为通过滤波器转换回的模拟信号}
    \label{fig:enter-label}
\end{figure}

\subsubsection{三角波形(Triangle Wave)}

\begin{itemize}
    \item \textbf{原始信号(左侧)}:图示为频率 $10 \text{ kHz}$,幅度 $5.00 \text{ V}$ 的理想三角波。
    \item \textbf{ADC转换的数字信号(中间)}:ADC采样率为 $100 \text{ kHz}$,远高于原始信号频率 $10 \text{ kHz}$。根据奈奎斯特采样定理,采样率 ($100 \text{ kHz}$) 远大于信号最高频率 ($10 \text{ kHz}$) 的两倍,因此ADC能够充分捕捉原始三角波的细节。 $8 \text{ bit}$ 的量化精度提供了 $2^8 = 256$ 个量化级别,使得PWM数字信号能够较好地反映原始信号的瞬时电压。从图中可以看出,PWM的占空比随着原始三角波的电压变化而平滑地调整,体现了较好的采样和量化效果。
    \item \textbf{重建波形(右侧)}:重建的模拟信号与原始三角波非常接近,表明ADC/DAC过程效果优秀。显示误差仅为 $0.078 \text{ V}$,信噪比(SNR)高达 $31.35 \text{ dB}$。这得益于高采样率有效地避免了混叠。DAC后的低通滤波器(LPF)截止频率为 $0.58 \text{ kHz}$,这对于平滑PWM信号(PWM频率 $0.8 \text{ kHz}$)非常有效。尽管输入信号频率为 $10 \text{ kHz}$,但重建后的波形保留了原始三角波的频率和基本形状,证明了在采样率足够高的情况下,即使LPF截止频率低于信号基频,也能在某种程度上重建出可辨识的波形,但这主要取决于PWM的载波频率与信号频率的关系以及LPF对这些频率成分的滤除效果。在这种高采样率下,PWM波形的平均值已经很好地反映了原始模拟信号。
\end{itemize}

\subsubsection{正弦波形(Sine Wave)}

\begin{itemize}
    \item \textbf{原始信号(左侧)}:图示为频率 $10 \text{ kHz}$,幅度 $5.00 \text{ V}$ 的理想正弦波。
    \item \textbf{ADC转换的数字信号(中间)}: $100 \text{ kHz}$ 的高采样率使得ADC能够精确地捕捉 $10 \text{ kHz}$ 正弦波的每一个周期内的多个采样点,从而在数字域精确地表示了原始信号。PWM信号的占空比随输入正弦波的瞬时电压变化而连续调整,波形脉冲密度也随之变化,呈现出明显的正弦波形状。
    \item \textbf{重建波形(右侧)}:重建的正弦波形与原始信号高度一致,其质量相当优秀。误差为 $0.267 \text{ V}$,SNR为 $22.43 \text{ dB}$。虽然误差略高于三角波,但从视觉上看,重建波形清晰地再现了 $10 \text{ kHz}$ 正弦波的频率和形状。LPF( $0.58 \text{ kHz}$ 截止频率)在去除PWM载波频率的同时,也保留了足够的信息来重建正弦波,这表明 $100 \text{ kHz}$ 的采样率产生了足够多的信息量,即使经过较低截止频率的LPF,也能通过平均值特性大致还原波形,尤其是在PWM频率 ($0.8 \text{ kHz}$) 高于LPF截止频率的情况下,LPF能有效滤除高频载波。
\end{itemize}

\subsubsection{方波形(Square Wave)}

\begin{itemize}
    \item \textbf{原始信号(左侧)}:图示为频率 $10 \text{ kHz}$,幅度 $5.00 \text{ V}$,占空比 $50\%$ 的理想方波。方波包含丰富的奇次谐波成分,其频谱带宽是无限的。
    \item \textbf{ADC转换的数字信号(中间)}:由于 $100 \text{ kHz}$ 的高采样率,ADC能够捕捉到方波的快速上升沿和下降沿附近的大量采样点。因此,PWM信号能够准确地在数字域表示方波的两个稳定电平以及转换过程,尽管由于量化效应和PWM的离散性,边沿看起来仍有阶梯状。
    \item \textbf{重建波形(右侧)}:重建的方波形误差为 $0.288 \text{ V}$,SNR为 $24.79 \text{ dB}$。重建信号与原始方波的形状非常接近,表现出良好的方波特征。尽管方波的边沿(上升沿和下降沿)由于LPF( $0.58 \text{ kHz}$ 截止频率)的低通特性而被平滑,无法达到理想方波的垂直跳变,但其高电平、低电平以及周期都得到了准确的重建。这再次证明了 $100 \text{ kHz}$ 的高采样率有效避免了混叠,并且DAC后的LPF成功地滤除了PWM载波成分,同时尽可能地保留了方波的主要特征。边沿的平滑是LPF的固有特性,因为方波的高频谐波被LPF滤除了,这在实际系统中是可接受的,并且通常可以通过更高截止频率的LPF或更高阶的DAC实现更陡峭的边沿。
\end{itemize}

\section{实验过程中遇到的问题及其解决办法}

\subsection{图形界面刷新效率低下}
\textbf{问题}:频繁重绘导致卡顿;
\textbf{原因}:未复用绘图对象;
\textbf{解决}:采用clear方法与局部刷新策略,避免阻塞。

\subsection{干扰检测灵敏度调整难题}
\textbf{问题}:误报或漏报多;
\textbf{原因}:归一化距离不准确;
\textbf{解决}:实测距离校准标准,调整阈值参数。

\subsection{滑块事件响应过于频繁}
\textbf{问题}:界面卡顿严重;
\textbf{原因}:使用<Motion>事件频率过高;
\textbf{解决}:改为<ButtonRelease>事件,或结合节流机制优化响应。

\subsection{中文字体显示乱码问题}
\textbf{问题}:图形中中文乱码;
\textbf{原因}:默认字体不支持中文;
\textbf{解决}:加载支持中文字体如SimHei,配置默认字体。

\subsection{多模块间数据同步问题}
\textbf{问题}:数据状态不同步;
\textbf{原因}:缺少集中管理;
\textbf{解决}:统一数据源,使用事件驱动同步机制,必要时引入锁机制。



\end{document}
